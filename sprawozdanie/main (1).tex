\UseRawInputEncoding
\documentclass{article}

\usepackage{float}
\usepackage{graphicx}
\usepackage{listingsutf8}
\usepackage[polish]{babel}
\usepackage[utf8]{inputenc}
\usepackage[T1]{fontenc}
\usepackage{lmodern}
\usepackage{geometry}
\usepackage{hyperref}
\usepackage{pgfplots}
\usepackage{array}
\usepackage{booktabs}
\usepackage{caption}
\usepackage{subcaption}
\usepackage{siunitx}
\sisetup{group-separator = {\,}}
\pgfplotsset{compat=1.18}
\usepackage{pgfplotstable}
\usepackage{xcolor}
\usepackage{csvsimple}  
\usepackage{etoolbox}    
\usepackage{xstring} 

\lstset{
  inputencoding=utf8,
  extendedchars=true,
  literate=
   {ą}{{\k{a}}}1 {Ą}{{\k{A}}}1
   {ć}{{\'c}}1   {Ć}{{\'C}}1
   {ę}{{\k{e}}}1 {Ę}{{\k{E}}}1
   {ł}{{\l}}1    {Ł}{{\L}}1
   {ń}{{\'n}}1   {Ń}{{\'N}}1
   {ó}{{\'o}}1   {Ó}{{\'O}}1
   {ś}{{\'s}}1   {Ś}{{\'S}}1
   {ź}{{\'z}}1   {Ź}{{\'Z}}1
   {ż}{{\.z}}1   {Ż}{{\.Z}}1,
  basicstyle=\ttfamily\small,
  numbers=left,
  numberstyle=\tiny,
  stepnumber=1,
  numbersep=5pt,
  showstringspaces=false,
  tabsize=2,
  breaklines=true,
  frame=single,
  language=C++,
  keywordstyle=\color{blue}\bfseries,
  commentstyle=\color{gray}\itshape,
  stringstyle=\color{teal},
}

\geometry{margin=2.5cm}
\title{Lista 1}
\author{Anna Leszczyńska}
\date{\ Październik 2025r }

\begin{document}
\maketitle
\section{Wstęp}
W sprawozdraniu porównane są algorytmy insertion sort, merge sort, heap sort oraz ich modyfikacje.
\section{Najciekawsze fragmenty kodu }

\subsection{Insertion sort: liczenie tylko zapisów do tablicy}
 W pętli przesuwającej elementy liczymy wyłącznie zapisy do tablicy (a nie przypisania do zmiennych lokalnych). To pozwala porównywać realny koszt pamięciowy między algorytmami.

 Pętla przesuwa większe elementy w prawo (\texttt{ASSIGN++} za każdy zapis), a test porządkowy podbija \texttt{COMP++}.
\begin{lstlisting}
while (j >= 0) {
    COMP++;                    
    if (!(a[j] > key)) break;
    a[j+1] = a[j]; ASSIGN++;   
    --j;
}
a[j+1] = key; ASSIGN++;
\end{lstlisting}

\subsection{DoubleInsertion: wstawianie pary bez nadpisywania}
 Każdą iterację traktujemy jako wstawienie  dwóch elementów naraz. Najpierw porządkujemy parę w 2 zmiennych (\(x\le y\)), potem wstawiamy najpierw \(x\), a następnie \(y\) do poszerzonego prefiksu — bez ryzyka nadpisania.

\begin{lstlisting}
int x = a[i-2], y = a[i-1];
COMP++; if (x > y) { int t=x; x=y; y=t; } 

int j = i - 3;
while (j >= 0) { COMP++; if (!(a[j] > x)) break;
  a[j+1] = a[j]; ASSIGN++; --j; }
a[j+1] = x; ASSIGN++;

int k = i - 2;
while (k >= 0) { COMP++; if (!(a[k] > y)) break;
  a[k+1] = a[k]; ASSIGN++; --k; }
a[k+1] = y; ASSIGN++;
\end{lstlisting}

\subsection{MergeSort(2): proste scalanie z jawnym kosztem buforów}
Podkreślamy, że kopiowanie do buforów to też zapisy do tablic. Dzięki temu „koszt pamięciowy” mergesorta jest porównywalny z heapsortem.

 Dwa bufory, liniowe scalanie, każdy zapis do \texttt{a[k]} zwiększa \texttt{ASSIGN}.
\begin{lstlisting}
for (int i = 0; i < n1; ++i) { Lbuf[i] = a[L+i]; ASSIGN++; }
for (int j = 0; j < n2; ++j) { Rbuf[j] = a[M+1+j]; ASSIGN++; }

while (i < n1 && j < n2) {
    COMP++; // Lbuf[i] <= Rbuf[j] ?
    if (Lbuf[i] <= Rbuf[j]) { a[k++] = Lbuf[i++]; ASSIGN++; }
    else                     { a[k++] = Rbuf[j++]; ASSIGN++; }
}
\end{lstlisting}

\subsection{MergeSort(3): wybór minimum z trzech w dwóch porównaniach}
 Wybieramy minimum z \(\{A[i],B[j],C[k]\}\) tylko dwoma porównaniami: najpierw \(A\) z \(B\), potem wynik z \(C\). Minimalizujemy \texttt{COMP}, kosztem większej liczby zapisów (trzy bufory).

\begin{lstlisting}
COMP++; bool Ai_le_Bj = (A[i] <= B[j]);
int vMinAB = Ai_le_Bj ? A[i] : B[j];
COMP++; 
if (vMinAB <= C[k]) {
    if (Ai_le_Bj) { a[t++] = A[i++]; ASSIGN++; }
    else          { a[t++] = B[j++]; ASSIGN++; }
} else {
    a[t++] = C[k++]; ASSIGN++;
}
\end{lstlisting}

\subsection{HeapSort (binarny)}
 Pokazujemy realny profil porównań: dwa potencjalne testy dzieci na poziom. Dzięki temu można porównać z kopcem trójdzielnym.
 Spychamy element w dół, każde porównanie zwiększa \texttt{COMP}, a zamiana to \texttt{countedSwap} (liczy tylko zapisy do tablicy).
\begin{lstlisting}
int l = 2*i+1, r = 2*i+2;
if (l < n) { COMP++; if (a[l] > a[largest]) largest = l; }
if (r < n) { COMP++; if (a[r] > a[largest]) largest = r; }
if (largest != i) { countedSwap(a, i, largest); i = largest; } else break;
\end{lstlisting}

\subsection{HeapSort (ternarny)}
Ten kod oblicza indeksy trzech potencjalnych dzieci elementu o indeksie $i$ ($3i+1, 3i+2, 3i+3$). Następnie, za pomocą porównań (COMP++), ustala, który z czterech elementów (rodzic lub jedno z trzech dzieci) jest największy. Jeśli to dziecko jest największe, element jest zamieniany z rodzicem (countedSwap inkrementuje ASSIGN++) i proces przesiewania w dół jest kontynuowany (i = largest). Jeśli rodzic jest największy, działanie funkcji jest przerywane 
\begin{lstlisting}
int c1 = 3*i+1, c2 = 3*i+2, c3 = 3*i+3;
if (c1 < n) { COMP++; if (a[c1] > a[largest]) largest = c1; }
if (c2 < n) { COMP++; if (a[c2] > a[largest]) largest = c2; }
if (c3 < n) { COMP++; if (a[c3] > a[largest]) largest = c3; }
if (largest != i) { countedSwap(a, i, largest); i = largest; } else break;
\end{lstlisting}
\section{Metodologia eksperymentów}
Program generuje pięć scenariuszy danych:
\texttt{Random}, \texttt{NearlySorted}, \texttt{Sorted}, \texttt{Reversed}, \texttt{FewUniques}.
Dla rozmiarów $n \in \{50, 200, 800, 2000\}$ i $T=5$ uruchamień dla każdego algorytmu, zapisuje wyniki do wyniki.csv w formacie:
scenario,n,algorithm,comps,assigns, time(w ms) .


\section{Wyniki: tabele}
Poniższe tabele są tworzone bezpośrednio z danych zawartych  w pliku wyniki.csv.

\begin{table}[H]
\centering
\csvreader[
  separator=comma,
  head to column names,
  tabular=l r l r r r,
  table head={\toprule Scen. & $n$ & Algorytm & Porównania & Przypisania & Czas [ms]\\\midrule},
  table foot={\bottomrule},
  late after line=\\,
  before reading=\setcounter{csvrow}{0}, 
  filter expr={ test{\ifnumless{\value{csvrow}}{12}} and test{\ifnumcomp{\n}{=}{50}} },
  after line=\stepcounter{csvrow}       
]{wyniki.csv}{}%
{\scenario & \n & \algorithm & \comps & \assigns & \timems}
\caption{Pierwsze 6 wyników dla $n=50$.}
\end{table}

\begin{table}[H]
\centering
\csvreader[
  separator=comma,
  head to column names,
  tabular=l r l r r r,
  table head={\toprule Scen. & $n$ & Algorytm & Porównania & Przypisania & Czas [ms]\\\midrule},
  table foot={\bottomrule},
  late after line=\\,
  before reading=\setcounter{csvrow}{0},
  filter expr={ test{\ifnumless{\value{csvrow}}{12}} and test{\ifnumcomp{\n}{=}{800}} },
  after line=\stepcounter{csvrow}
]{wyniki.csv}{}%
{\scenario & \n & \algorithm & \comps & \assigns & \timems}
\caption{Pierwsze 6 wyników dla $n=800$.}
\end{table}

\begin{table}[H]
\centering
\csvreader[
  separator=comma,
  head to column names,
  tabular=l r l r r r,
  table head={\toprule Scen. & $n$ & Algorytm & Porównania & Przypisania & Czas [ms]\\\midrule},
  table foot={\bottomrule},
  late after line=\\,
  before reading=\setcounter{csvrow}{0},
  filter expr={ test{\ifnumless{\value{csvrow}}{12}} and test{\ifnumcomp{\n}{=}{2000}} },
  after line=\stepcounter{csvrow}
]{wyniki.csv}{}%
{\scenario & \n & \algorithm & \comps & \assigns & \timems}
\caption{Pierwsze 6 wyników dla $n=2000$.}
\end{table}


% ====================== WYKRES 1: Przypisania (Random, n=800) ======================
\def\dataAssigns{}%
\csvreader[
  separator=comma,
  head to column names,
  filter test = \ifboolexpr{
     test{\IfStrEq{\scenario}{Random}} and test{\ifnumcomp{\n}{=}{800}}
  }
]{wyniki.csv}{}{
  \xdef\dataAssigns{\dataAssigns(\algorithm,\assigns)}%
}

\begin{figure}[H]
\centering
\begin{tikzpicture}
\begin{axis}[
  ybar,
  width=.95\linewidth, height=7cm,
  symbolic x coords={Insertion,DoubleInsertion,MergeSort2,MergeSort3,HeapBinary,HeapTernary},
  xtick=data,
  xticklabel style={rotate=20, anchor=east},
  ylabel={Przypisania (zapisy do tablic)},
  title={Random, n=800},
  ymin=0,
  bar width=8pt
]
\addplot+ coordinates {\dataAssigns};
\end{axis}
\end{tikzpicture}
\caption{Zapis do tablic (Random, n=800).}
\end{figure}

\section{Wnioski}
\begin{itemize}
 \item \textbf{Insertion} wymaga największej ilości przypisań i zajmuje najdłużej.
 \item \textbf{Heap (binarny) vs Heap (ternarny):} Różnice w liczbie porównań i zapisów są niewielkie, zależne od danych.
 \item \textbf{Rodziny Merge vs Heap:} MergeSort wymaga dodatkowych buforów, więc liczba zapisów do tablic jest wyższa niż w HeapSort.
\end{itemize}

\end{document}
